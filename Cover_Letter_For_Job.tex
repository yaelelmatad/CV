
%%%%%%%%%%%%%%%%%%%%%%%%%%%%%%%%%%%%%%%%%
% Thin Formal Letter
% LaTeX Template
% Version 1.11 (8/12/12)
%
% This template has been downloaded from:
% http://www.latextemplates.com
%
% Original author:
% WikiBooks (http://en.wikibooks.org/wiki/LaTeX/Letters)
%
% License:
% CC BY-NC-SA 3.0 (http://creativecommons.org/licenses/by-nc-sa/3.0/)
%
%%%%%%%%%%%%%%%%%%%%%%%%%%%%%%%%%%%%%%%%%

%----------------------------------------------------------------------------------------
%	DOCUMENT CONFIGURATIONS
%----------------------------------------------------------------------------------------

\documentclass{letter}


\usepackage{graphicx}
\usepackage{amsmath}
\usepackage{amsfonts}
\usepackage{amssymb}
\usepackage{epstopdf}
\usepackage{ulem}
\usepackage[draft]{pdfpages}

% Adjust margins for aesthetics
%\addtolength{\voffset}{-1.5in}
%\addtolength{\hoffset}{-1.3in}
%\addtolength{\textheight}{2cm}

%\longindentation=0pt % Un-commenting this line will push the closing "Sincerely," to the left of the page

%----------------------------------------------------------------------------------------
%	YOUR NAME & ADDRESS SECTION
%----------------------------------------------------------------------------------------



%\signature{Yael S. Elmatad} % Your name for the signature at the bottom

\address{New York University \\ Department of Physics \\ 4 Washington Place, Room 610 \\ New York, NY 10012 \\ (212) 998-7720} % Your address and phone number

%----------------------------------------------------------------------------------------

\begin{document}


%----------------------------------------------------------------------------------------
%	ADDRESSEE SECTION
%----------------------------------------------------------------------------------------

\begin{letter}{Faculty Search Committee \\
Physics Department, MR 419 \\
City College of New York  \\
160 Convent Avenue, New York, NY 10031} % Name/title of the addressee

%----------------------------------------------------------------------------------------
%	LETTER CONTENT SECTION
%----------------------------------------------------------------------------------------

\opening{\textbf{Dear Search Committee,}}
 
I am writing to apply for the tenure-track position in the Physics Department in theoretical condensed matter physics.  I am currently a Faculty Fellow/Postdoctoral Fellow at New York University in the Center for Soft Matter Research in the Physics Department.  This is an independent position with a research budget and teaching responsibilities intended to promote diversity in the university. Previously, I was a graduate student with David Chandler at the University of California, Berkeley, receiving my Ph.~D. in Physical Chemistry in August 2011.  I graduated from New York University in 2006 with a B.S. in Chemistry with minors in Mathematics and Computer Science.  My main academic background is in the field of computational and theoretical statistical mechanics and thermodynamics with an emphasis on slow (supercooled, dynamically heterogeneous) systems and rare events.

%I am writing to apply for the tenure-track position in theoretical/computational biophysics in the Laboratory of Chemical Physics, NIDDK, NIH recently advertised in Science Careers. I am currently a Distinguished Postdoctoral Fellow in the California Institute of Quantitative Biosciences (QB3) at the University of California, Berkeley. This is an independent position with an annual research budget, awarded to those intending to pursue a vigorous and interdisciplinary research career in quantitative biology. Previously, I was a postdoctoral researcher in the Department of Chemistry at Stanford University, in the laboratory of Vijay Pande. Prior to that, I was a graduate student with Ken Dill at UCSF, receiving my Ph.D. in Biophysics in December of 2006. I graduated from Caltech with a B.S. in Biology in 1999.

My research focus is two pronged: I am interested in slow out-of-equilibrium systems with a main focus on glassy dynamics as well as developing new computational methods to study such out-of-equilibrium systems eventually hoping to apply those methods towards engineering new materials.  My expertise is in glassy dynamics and I remain involved in understanding the role of dynamic facilitation in supercooled liquids and granular material.  I am further interested in the way scientists measure glassy dynamics -- specifically related to the role of probe particles as molecular thermometers.  Do these probes alter the dynamics of the host material?  Can their use be extended to measurements at different length scales?  Beyond this, I am interested in developing new computational methods to study rare events in such glassy (and other dynamically interesting) systems.  Through this development, I hope to be able to make recommendations for designing new materials that have certain desirable dynamical phenomena by exploring model parameter space.

I am also interested in harnessing the power of new computational tools available to computational scientists. I am interested in using new hardware advances -- such as running simulations on graphics card processors rather than conventional CPUs.  To that end, I believe it is important to incorporate the latest available hardware and algorithms into scientific research to make the most of our current tools.

I believe my research would be a good fit for the CCNY Physics department. As a chemical physicist, I believe I would be thrilled to work closely with the researchers in the Levich institute, as my research interests span from physics to chemistry and from biophysics to material science.  

\vspace{2\parskip} % Extra whitespace for aesthetics
\closing{Sincerely,
\vspace{2\parskip} % Extra whitespace for aesthetics

\fromsig{\includegraphics[scale=0.2]{signature.jpg}} \\ 
\fromname{Yael Elmatad}
}


%\ps{P.S. You can find additional information attached to this letter.} % Postscript text, comment this line to remove it

\encl{Research Statement, Teaching Statement, CV - 13 pages total} % Enclosures with the letter, comment this line to remove it

%----------------------------------------------------------------------------------------

\end{letter}
 
\end{document}