\documentclass[11pt]{article}
\usepackage[left=1.9cm,top=2cm,right=1.9cm,bottom=1.9cm]{geometry}
\geometry{letterpaper}                   % ... or a4paper or a5paper or ... 
%\geometry{landscape}                % Activate for for rotated page geometry
%\usepackage[parfill]{parskip}    % Activate to begin paragraphs with an empty line rather than an indent
\usepackage{graphicx}
\usepackage{amsmath}
\usepackage{amsfonts}
\usepackage{amssymb}
\usepackage{epstopdf}
\usepackage{ulem}
\usepackage[draft]{pdfpages}
\DeclareGraphicsRule{.tif}{png}{.png}{`convert #1 `dirname #1`/`basename #1 .tif`.png}
\def \dbar{{\mathchar'26\mkern-12mu d}}
\newcommand{\kB}{k_{\mathrm{B}}}
\newcommand{\traj}{\text{traj}}
\newcommand{\ee}{\mathrm{e}}
\newcommand{\tobs}{t_\mathrm{obs}}
\newcommand{\eps}{\epsilon}
\newcommand{\sig}{\sigma}
\newcommand{\ii}{\mathrm{i}}
\newcommand{\CC}{\mathcal{C}}
\newcommand{\WW}{\mathbb{W}}
\newcommand{\HH}{\mathbb{H}}
\newcommand{\EE}{\mathbb{E}}
\newcommand{\FF}{\mathcal{F}}

\newcommand{\rate}{\lambda}
\newcommand{\upr}{\gamma}

\newcommand{\zz}{\ee^{-s}}
\newcommand{\zzs}{\ee^{-s^*}}

\newcommand{\Sc}{S}

\renewcommand{\figurename}{\textbf{Figure}}

% Different font in captions
\newcommand{\captionfonts}{\small}

\makeatletter  % Allow the use of @ in command names
\long\def\@makecaption#1#2{%
  \vskip\abovecaptionskip
  \sbox\@tempboxa{{\captionfonts #1: #2}}%
  \ifdim \wd\@tempboxa >\hsize
    {\captionfonts #1: #2\par}
  \else
    \hbox to\hsize{\hfil\box\@tempboxa\hfil}%
  \fi
  \vskip\belowcaptionskip}
\makeatother   % Cancel the effect of \makeatletter

\renewcommand{\abstractname}{Summary}

 \def\dsp{\def\baselinestretch{2.0}\large\normalsize}
%\dsp

\title{{Teaching Statement}}
\author{Yael S. Elmatad}
\date{}                                           % Activate to display a given date or no date
\begin{document}
\maketitle
%\section{}
%\subsection{}
%\includepdf[pages=-]{pdfpages.pdf}
%\section{Long-Term Objectives}

\section{Teaching Background}

One of the selling points of my current position as an NYU Diversity Faculty Fellow/Postdoctoral Fellow was the ability to combine my love of independent research with my sense of teaching responsibility.  Over the past few semesters, I have enjoyed my time as the instructor of two required upper level courses: Thermal Physics and Quantum Mechanics I.  Throughout my time, I have had to grapple with the pitfalls that often face new faculty such as writing fair examinations and explaining material in a novel way -- not to mention dealing with a wide variety of student concerns ranging from questions regarding difficult material to issues concerning personal matters.  Designing these courses myself has been challenging but rewarding.  In spite of these challenges, I was pleased that 11 out 12 students said they would recommend me as an instructor to a friend. 

Up until my postdoctoral work, my education and teaching experience has been mostly in chemistry departments.  However, my education has always been on the very physical side of chemistry -- opting for extra physics, mathematics, and computer science coursework rather than pure chemistry.  As a sophomore undergraduate I was already leading general chemistry clinic sections.  By my junior year I was one of the two Physical Chemistry teaching assistants, teaching both semesters (quantum mechanics, statistical mechanics, thermodynamics, gas kinetics, etc...) while still enrolled in a variety of upper level coursework.

In graduate school at UC Berkeley, I was a teaching assistant for three semesters.  My first semester, I was selected out of all the incoming graduate students to teach the laboratory portion of the General Chemistry for majors (rather than non-majors). I then went on to teach the upper level Physical Chemistry Thermodynamics/Statistical Mechanics course.  For my work as a teaching assistant in that course, I received a Graduate Student Instructor Teaching Award from the university and received reviews stating that I was the best teaching assistant my students encountered during their studies.  In my fourth year, I was selected to be the teaching assistant for the graduate level statistical mechanics course in the chemistry department.

\section{Teaching Tools and Philosophy}

Throughout my teaching work, I have endeavored to make the material both accessible and relatable.  One strategy that I have used is to include modern research into my teaching assignments.  In statistical physics, I must go through all the different kinds of engines -- something that seems archaic to most students (and instructors!).  However, a recent article in Nature Physics about a one-particle Stirling engine shows how even this mundane topic can be made interesting and is on the cutting edge of scientific discovery.  In my Quantum I class I introduced the Einstein-Rosen-Podolsky paradox, Bell's inequality, and highlighted several papers (all within the past year) on quantum entanglement and quantum communication.

Moreover, as a computational chemical physicist, I know the power of computing in modern science.  Students who can harness computational tools have an advantage over those who shy away from it  -- whether they choose to go into theory \textit{or} experiment.  As a graduate teaching assistant at Berkeley, I was charged with introducing my students to the basics of programming (in C++) so they could learn to run simple Monte Carlo simulations.  This spring, I intend to do the same with my undergraduate physics students only this time I will teach them Python, a more modern language.

I believe integrating modern research and techniques (such as computer simulation) makes my courses more enriching to students who realize that what I am teaching them allows them access to some of the most current ways of looking at the world, and not only material stuck in the past.

Furthermore, as a woman in science, I believe that it is deeply important to have female role models.  As a high school student, my first chemistry teacher was a woman and I believe that had an important impact on my future trajectory.  At NYU and UC Berkeley, however, I did not have a single female professor in any of my chemistry, physics, mathematics, or computer science courses.  Fortunately, I did have a female academic undergraduate advisor who pushed me and believed in my success.  In graduate school, I became a member of the women in chemistry organization (Iota Sigma Pi) -- a group dedicated to shrinking the gender gap in science.  As a postdoctoral fellow, I have been volunteering with the undergraduate Women in Science group, acting as an an additional mentor for next batch of budding scientists.  I am keenly aware that my ability to express confidence and competence as an instructor and mentor has a lasting impact on future generations of scientists.  To some, it expresses that they too can be in my shoes, while to others it highlights that people from varied backgrounds are just as capable as anyone else.

\section{Future} %section to edit depending on application

My diverse educational background in physical sciences makes me suitable for teaching a wide variety of courses across several departments.  My favorite courses are those close to my research background such as those that deal with Thermodynamics, Statistical Mechanics, and Mathematical Methods.  However, my experience and teaching also makes me suitable for other courses such as General Chemistry and Physics, Quantum Mechanics, Biophysics, Condensed Matter, and Classical Mechanics.

Moreover, I would love the opportunity to design new courses.  For example, one idea I have been considering is a course on non-equilibrium statistical mechanics and simulation methods that includes current work on out-of-equilibrium and dynamical systems such as the Jarzynski inequality and Crooks fluctuation theorems as well as rare event path sampling methods developed to study dynamical processes and slow systems.  Additionally, I believe it is important for undergraduates to be exposed to state of the art research and I believe a course that focuses on reading modern and current literature analytically -- that is understanding the material, critiquing it, as well as suggesting future steps in that line of work -- would highlight another facet of being a scientist normally excluded from undergraduate education.

I also look forward to mentoring graduate and undergraduate students in my new position as a research group leader.  I was fortunate enough to have an excellent graduate advisor at UC Berkeley who was able to manage his group effectively without micromanaging us -- thus allowing us the ability to discover things independently, though still under his guidance.  I hope to bring this balance to my own group, knowing when to push and be assertive, but also knowing when to back off and allow for individual discovery.

\section{At CUNY} %section to edit depending on application

At an undergraduate level, I could teach the majority of physics courses for both majors and non-majors such as Ideas in Physics, General Physics, Elementary Physics, any of the Honors courses, any of the Quantum Mechanics courses (such as the QM for Applied Engineers), Biophysics, Mechanics I \& II, Thermodynamics and Statistical Physics, Kinetic Theory and Statistical Mechanics, The Physics and Chemistry of Materials, The Physical Universe, Science I \& II, Development and Analysis of Ideas in Classical Science, as well as Development and Analysis of Ideas in Contemporary Science.

At the graduate level, I would be more than comfortable teaching any level of Statistical Mechanics, Quantum Mechanics I, Mathematical Methods in Physics, Computational Methods in Physics, and Biophysics.

\end{document}  
